\defcitealias{ali_et_al2015}{A15}

\chapter{Conclusion}
\label{c.conclusion}

The work presented in this thesis can be summarized by the following:

\begin{itemize}
\item The original 21\,cm power spectrum results from the 64-element configuration of PAPER, first presented in \citetalias{ali_et_al2015}, have been found to suffer from cosmological signal loss associated with the use of empirical covariances. Qualitative and quantitative investigations of the origin of this loss have shaped a deeper understanding of why signal loss arises, how to mitigate it, and how data analysis choices affect loss and power spectrum sensitivities. From this work emerges a new method to quantify signal loss which sets a standard across the field as a whole in terms of performing detailed calculations and assessments of loss.
\item Numerous other power spectrum analysis errors that affect both \citet{parsons_et_al2014} and \citetalias{ali_et_al2015} have been discovered and revised. These errors mostly concern error estimation techniques used by PAPER. While both the empirical and theoretical errors were underestimated in the original analyses, updated methods are used to produce the power spectrum results throughout this work. At a broader level, this work has influenced the development of additional error estimation techniques for both PAPER and HERA analyses in a push for more robust tests and validation.
\item Using a simplified pipeline and incorporating all updated methods, PAPER-64 places new 21\,cm upper limits across a range of redshifts that are competitive with results from other experiments. Most notably, novel components of this analysis include investigations of PAPER's redundancy, comparisons with sky simulations, and analyses of jackknife tests.
\item The 128-element configuration of PAPER has been found to present unique challenges in terms of data quality. Algorithms have been developed in order to locate and remove contaminated data, which are now standard quality-check measures that have been incorporated into HERA's real-time processing system. However, because the quality of PAPER-128 data is poor compared to that of PAPER-64, we only present rough power spectrum results for two epochs of data, which are found to be competitive with the revised PAPER-64 results.
\item Initial HERA analyses are ongoing and signal loss characterization continues in an effort to build intuition behind our power spectrum analysis choices. Preliminary work suggests that the shape and accuracy of an eigenspectrum of a covariance is closely related to the amount of signal loss that is incurred when weighting data by the covariance. Future work in characterizing eigenspectra will help in understanding the interplay between convergences, weightings, and signal loss.
\end{itemize}

Taken as individual parts, the work in this thesis represents a collection of subtle lessons concerning 21\,cm power spectrum estimation. But looked at as a whole, it tells a bigger story that serves as a reminder of the uncertainty, challenges, and unpredictability of science in general. What began as a promising future for PAPER-128 (following a field-leading result from PAPER-64) led to the discovery of errors in PAPER-64 and subsequently an unanticipated retraction and revision. However, it is important to remember that good science --- regardless of the final outcome --- is science that is honest, careful, repeatable, and communicated. The broader story told in this thesis is, in some ways, one that is very common (but not often talked about) in science. It is clear that the issues we have found are not unique to any one experiment and, although unexpected, are pushing scientific progress in a positive way. As we close the chapter on PAPER, the lessons we have learned will continue to influence the path ahead, as unpredictable as it may be. What is predictable, though, is that we are now equipped with better tools and deeper understandings, and the field of 21\,cm cosmology has a lot to look forward to. 




